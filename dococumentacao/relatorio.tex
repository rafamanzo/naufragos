\documentclass[a4paper,12pt]{article}

\usepackage{url}
\usepackage[brazil]{babel}
\usepackage{hyperref} 

\author{ }
\title{N\'aufragos}

\begin{document}

\maketitle

\section{Autores}
Douglas Bettioli Barreto (NUSP 6920223)

Giancarlo Rigo (NUSP 6910034)

Rafael Reggiani Manzo (NUSP 6797150)
\section{Fases anteriores}
\label{sec:fasesanteriores}
\subsection{Primeira fase}
\subsubsection{Relatorio}
- Programa

Como executar: ./Naufragos framesPorSegundo tempoMaximo, por exemplo, ./Naufragos 10 10
Como compilar: make

- Testes

Como executar: ./testes 
Como compilar: make testes

- Limpeza

Dos códigos- objetos: 	      make clean
Dos arquivos temporários:     make clean-temp
Remoção do conteúdo da pasta: make real-clean

- Relatório dos Testes:

-> Robustez: O programa sobrevive em condições especiais, como número de passageiros muito grande ou muito pequeno ou velocidade muito próxima de zero? O que acontece se dois passageiros se chocam um com o outro? 

	Com número de passageiros muito pequeno, o programa roda sem problema. Já com número de passageiros muito grande, o resultado varia devido a iniciação do rand. Entretanto, para número de passageiros maiores de 300, o programa na maioria dos casos travou.
	As velocidades muito baixas, as pessoas na maioria dos casos, param devido ao trucamento(int).
	Ao ocorrer uma colisão será impresso 'X' no local onde ocorreu a mesma.

-> Corretude: O programa detecta corretamente o choque entre 2 passageiros?
	 A análise de corretude de colisões será baseada no método corretude que cria um mar pequeno com muitas pessoas, ocasionando varias colisões e 
mostrando que o programa resiste e realiza corretamente o choque entre os 2 passageiros.

subsubsection{Comentarios do Monitor}
O relatório entregue está ok, com uma avaliação honesta do que ocorre nos testes. Porém, seria bom se vocês comentassem melhor o código (em especial as funções, o que elas fazem, o que recebem e o que devolvem). O código também precisa ser melhor estruturado, com a utilização de arquivos headers.

Legal parametrizar o programinha com argumentos, porém seria interessante exibir um "help" quando o programa é chamado sem argumentos ou com parâmetros incorretos. Nesse mesmo help deixem claro quais são os valores mínimo e máximo aceito por cada parâmetro.

Já os testes automatizados não funcionam, aparentemente foram implementados às pressas (o switch, por exemplo, chama sempre "robustez"). Uma das possíveis razões do mau funcionamento para instâncias grandes é que a ED está vazando MUITA memória; vocês precisam corrigir isso.

OBS: para o próximo EP, por favor enviem somente um zip, através da conta Moodle de somente um dos membros do grupo.

\subsection{Segunda Fase}
\subsubsection{Relat\'orio}
- Programa

Como executar:
./Naufragos para executar com valores padrao ou
./Naufragos arg1 arg2 arg3 arg4 arg5, no qual 
	arg 1 - Frames por Segundo
	arg 2 - Tempo de Execucao do Programa em Segundos
	arg 3 - Frequencia de Criacao de Pessoas
	arg 4 - Velocidade Media de Criacao de Pessoas
	arg 5 - Numero de Recifes
Exemplo: ./Naufragos ou ./Naufragos 20 10 1 4 10

Como compilar: make

- Problemas conhecidos a serem eliminados na proxima fase

Nesta versao nao compilamos testes automatizados. Mas observando o funcionamento do programa em algumas situacoes chegamos as seguintes conclusoes:
 * O programa aguenta algo em torno de 200 pessoas, 20 corais, o asimov e 2 botes ao mesmo tempo num mar de 1024x768. Nao por que a atualizacao fique muito lenta, mas por que quando os botes e pessoas tem que ser gerados novamente, nao ha espaco nas bordas do mar. Ou seja, o mar fica poluido. Setar um limite de pixels ocupados no mar seria o correto.
 * Tudo tem forma circular pela facilidade de detectar colisoes com esta forma. Na proxima fase as figuras serao o mais proximas de circulos quanto for possivel.
 * O Allegro nao se da muito bem com o Valgrind. Testando o programa sem suas funcoes nao houveram probelmas. Agora executando com as funcoes do Allegro ele acusa muitos leaks e variaveis nao inicializadas. Mas, com o teste sem o Allegro, conclui-se que as EDs certamente nao vazao mais memoria.

\subsubsection{Coment\'arios do monitor}
A maneira como vocês fizeram para controlar o FPS pode gerar atrasos. O certo é, para cada iteração do loop principal: 1- marque o tempo inicial; 2- faça a iteração do joguinho; 3- baseado no tempo inicial, mande dormir).

Na parte de estruturação, vou pedir que vocês "enxuguem" o main, colocando o loop principal do jogo em uma função de uma biblioteca a parte. Deem uma olhada no capítulo 3 deste tutorial, para entenderem como geralmente um joguinho é gerenciado pela função main: http://www.glost.eclipse.co.uk/gfoot/vivace/vivace.txt

Outra coisa, ainda sobre estruturação: deixem os headers junto com os .c. Ainda sobre os headers, utilizem a diretiva ifndef. Também não utilizem "números mágicos" no meio do código: por exemplo, a resolução do modo gráfico é claramente um parâmetro que deveria ser uma diretiva em um header, ao invés de arbitrado diretamente sempre que utilizado.

Por fim, documentem adequadamente o código: em todos os headers, digam o que cada função recebe, o que faz e o que devolve.

\section{Ultima Fase}
\subsection{Configurador}
O arquivo de configura\c c\~ao encontra-se em configurador/config.sys e segue a seguinte gram\'atica:

\begin{verbatim}
tt -> tamanho de tela
tt=<COMPRIMENTO>;<ALTURA>;

nj -> nome de jogador
nj=<NOME>;

hg -> marcador de highscore
hg=<NOME>;<PONTUAÇÃO>;

nv -> numero de vidas iniciais
nv=<VALOR>;

vmax -> velocidade maxima
vmax=<VALOR>;

vmin ->velocidade minima
vmin=<VALOR>;

fcp -> frequencia de criação de peossos
fcp=<VALOR>;

vcp -> velocidade de criação de pessoas
vcp=<VALOR>;

nr -> numero de recifes
nr=<VALOR>;

di -> dificuldade inicial
di nv=<VALOR>; vmax=<VALOR>; .... 
\end{verbatim}

\subsection{Novas funcionalidades}
Nesta vers\~ao foram implementadas figuras bitmap, toda a jogbilidade como limite de pessoas no bote, ancoragem, etc.
Esta \'e a parte vis\'ivel, mas no programa foram criadas estruturas de dados separadas para pessoas, botes e objetos estaticos. Os bugs conhecidos da vers\~ao anterior foram corrigidos.

\subsection{Problemas conhecidos}
Infelizmente, faltou implementar os highscores.

\subsection{Como jogar}
\subsubsection{Bote 1}
\begin{verbatim}
Propulsao: w
Freio: s
Girar horario: d
Girar anti-horario: a
Ancorar: espaco
\end{verbatim}
\subsubsection{Bote 2}
\begin{verbatim}
Propulsao: up arrow
Freio: down arrow
Girar horario: right arrow
Girar anti-horario: left arrow
Ancorar: ENTER
\end{verbatim}
\end{document}
